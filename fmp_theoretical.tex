% fmp_theoretical.tex — The Fractal Metascience Paradigm: Theoretical Core
\documentclass[11pt,a4paper]{article}
\usepackage[utf8]{inputenc}
\usepackage[T1]{fontenc}
\usepackage{lmodern}
\usepackage{amsmath,amssymb}
\usepackage{graphicx}
\usepackage{hyperref}
\usepackage{caption}
\usepackage{geometry}
\geometry{margin=1in}
\title{The Fractal Metascience Paradigm: Foundations, Models, and Implications for Complex Knowledge Systems}
\author{Abdurashid Abdukarimov \\ Independent Researcher, Tashkent, Uzbekistan \\ ORCID: 0009-0000-6394-4912}
\date{2025}
\begin{document}
\maketitle
\begin{abstract}
The Fractal Metascience Paradigm (FMP) introduces a transformative epistemological framework for 21st-century science. It proposes that knowledge, like nature itself, is fractal—self-similar, recursive, and emergent across multiple scales of observation. Moving beyond reductionism and linear causality, FMP integrates insights from complexity science, quantum cognition, and recursive epistemology to establish a unified model of knowledge creation and organization.
\end{abstract}
\section{Introduction}
% ... (полный текст твоей теоретической статьи из документа)
\begin{figure}[ht]
\centering
\includegraphics[width=0.6\textwidth]{fractal_schema_1.png}
\caption{Fractal epistemic schema: recursive layers L0–L7.}
\end{figure}
\bibliographystyle{apalike}
\bibliography{fmp_theoretical}
\end{document}