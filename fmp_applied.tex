% fmp_applied.tex — Applied Validation: Terra Codex and AIUZ Ecosystem
\documentclass[11pt,a4paper]{article}
\usepackage[utf8]{inputenc}
\usepackage[T1]{fontenc}
\usepackage{lmodern}
\usepackage{amsmath,amssymb}
\usepackage{graphicx}
\usepackage{hyperref}
\usepackage{caption}
\usepackage{geometry}
\geometry{margin=1in}
\title{Applied Validation of the Fractal Metascience Paradigm: The Terra Codex and AIUZ Ecosystem Case Studies}
\author{Abdurashid Abdukarimov \\ Independent Researcher, Tashkent, Uzbekistan}
\date{2025}
\begin{document}
\maketitle
\begin{abstract}
The validation of the Fractal Metascience Paradigm (FMP) requires not only theoretical coherence but also demonstrable, repeatable evidence of its operational viability. To this end, several large-scale implementations—most notably the Terra Codex and AIUZ Ecosystem—serve as empirical laboratories for the application of fractal epistemology in artificial intelligence, education, and knowledge system design.
\end{abstract}
\section{Applications and Validation Paradigms}
% ... (полный текст прикладной статьи)
\begin{figure}[ht]
\centering
\includegraphics[width=0.6\textwidth]{human_ai_symbiosis.png}
\caption{Human–AI symbiosis cycle in Terra Codex deployments.}
\end{figure}
\bibliographystyle{apalike}
\biblio graphy{fmp_applied}
\end{document}